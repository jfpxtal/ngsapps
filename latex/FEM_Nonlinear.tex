\documentclass[11pt,a4paper]{article}

\usepackage{a4wide}
\usepackage{graphicx}
\usepackage{latexsym}
\usepackage{epsfig}
\usepackage{amssymb}
\usepackage{amstext}
\usepackage{amsgen}
\usepackage{amsxtra}
\usepackage{amsgen}
\usepackage{amsthm}
\usepackage{color}

%\usepackage{showkeys}

\usepackage{subfigure}

\def\eps{\varepsilon}

\def\d{\,{\rm d}}
\def\div{{\rm div}}

\def\RR{\mathbb{R}}
\def\CC{\mathbb{C}}

\newcommand\diam{\mathop{\rm diam}}

\newtheorem{thm}{Theorem}[section]
\newtheorem{prop}[thm]{Proposition}
\newtheorem{lemma}[thm]{Lemma}
\newtheorem{cor}[thm]{Corollary}
\newtheorem{rem}[thm]{Remark}
\newtheorem{imprem}[thm]{Important remark}

\theoremstyle{definition}
\newtheorem{example}[thm]{Example}
\newtheorem{definition}[thm]{Definition}
\newtheorem{algorithm}[thm]{Algorithm}
\newtheorem{conjecture}{Conjecture}
\newtheorem{assumption}[thm]{Assumption}
\newtheorem{scheme}{Scheme}



\numberwithin{equation}{section}

%\renewcommand{\baselinestretch}{2}

\newcounter{hypcounter}

\newcommand{\N}{\mathbb{N}}
\newcommand{\Z}{\mathbb{Z}}
\newcommand{\R}{\mathbb{R}}
% \newcommand{\RR}{\mathbb{R}}
\newcommand{\C}{\mathbb{C}}
\newcommand{\ol}[1]{\overline{#1}}
\newcommand{\ul}[1]{\underline{#1}}
\newcommand{\pf}[2]{\frac{\partial {#1}}{\partial {#2}}}
%\newcommand{\ps}[2]{\frac{\partial^2 {#1}}{\partial {#2}^2}}
\newcommand{\df}[2]{\frac{d {#1}}{d {#2}}}
%\newcommand{\ds}[2]{\frac{d^2 {#1}}{d {#2}^2}}
\newcommand{\dx}{\Delta x}
\newcommand{\dt}{\Delta t}
\newcommand{\eqdef}{:=}
\newcommand{\dv}{\text{div}}
\newcommand{\di}{\displaystyle}
\newcommand{\lk}{\left(}
\newcommand{\rk}{\right)}

\newcommand{\energy}{{\cal E}}
\newcommand{\x}{{\bf x}}
% \newcommand{u}{u}
\newcommand{\q}{\quad}
\newcommand{\na}{\nabla}

\newcommand{\weak}{\rightharpoonup}
\newcommand{\embedded}{\hookrightarrow}

\newcommand{\bigO}{{\cal O}}
\newcommand{\smallO}{{\cal o}}
\newcommand{\dom}{{\cal D}}
\newcommand{\norm}[1]{\Vert #1 \Vert}
\newcommand{\abs}[1]{\vert #1 \vert}
\newcommand{\intxt}[1]{\int\limits_0^{t_*} \int\limits_0^1 #1 ~dx dt}
\newcommand{\intx}[1]{ \int\limits_0^1 #1 ~dx }
\newcommand{\intt}[1]{\int\limits_0^{t_*} #1 ~dt}
\newcommand{\set}[2]{\{~#1~|~#2~\}}
\newcommand{\M}{{\cal M}}
\newcommand{\A}{{\cal A}}
\newcommand{\norma}{\|\hspace{-0.4mm}|}
\newcommand{\jump}[1]{[\![#1]\!]}
\newcommand{\avg}[1]{\{\!\{#1\}\!\}}

\newcommand{\ur}{\underline{r}}
\newcommand{\ub}{\underline{b}}
\newcommand{\rr}[1]{\textcolor{red}{#1}}
\newcommand{\rb}[1]{\textcolor{blue}{#1}}
\DeclareMathOperator*{\argmin}{arg\,min}
% \DeclareMathOperator*{\liminf}{lim\,inf}

%opening
\title{Questions regarding the FEM analysis of cross-diffusion / nonlocal equations with degenerate diffusion}
\author{Jan-F. Pietschmann}

\begin{document}
\maketitle
\subsubsection*{Introduction}
For $x\in\Omega\subset \RR^N$ open and bounded. and $t>0$, we are interested in following \textbf{two problems}:
\begin{align}\label{eq:porousfem}
\mathrm{(P1)}\quad\partial_t \rho = \nabla\cdot (\rho [ \eps\nabla \rho^{m-1} + \nabla K \ast \rho]),
\end{align}
$m\in (1,\infty)$, supplemented with the initial conditions
\begin{equation}
	\rho(\cdot,0) = \rho_0\text{ in } \Omega.
\end{equation}
And the system of cross-diffusion equations for the densities $r=r(t,x)$ and $b=b(t,x)$ given by
\begin{equation*}
    \mathrm{(P2)}\quad\begin{aligned}
 \partial_t r &= \nabla \cdot (D_1[(1-\rho)\nabla r + r \nabla \rho ] + r(1-\rho)\nabla V),\\
\partial_t b &= \nabla \cdot (D_2[ (1-\rho)\nabla b + b \nabla \rho ] + b(1-\rho)\nabla W),   \end{aligned}    
\end{equation*}
where $r,\,b$  $\rho = r+b$ and supplemented with the initial conditions
\begin{equation}
	r(\cdot,0) = r_0, \quad b(\cdot,0) = b_0 \text{ in } \Omega.
\end{equation}
Note that alternatively, we can write the system in the (less symmetric) form
 \begin{equation}
    \begin{aligned}
 \partial_t r &= \nabla \cdot (D_1(1-b)\nabla r + r \nabla b + r(1-\rho)\nabla V]),\\
 \partial_t b &= \nabla \cdot (D_2[ (1-r)\nabla b + b \nabla r + b(1-\rho)\nabla W]).
\end{aligned}    
\end{equation}
\subsubsection*{Connection between the two problems}
First of all, note that in the case $D_1=D_2=1$ and $V=W$, we can add the two equations in (P1) and obtain
\begin{align}\label{eq:sum}
 \partial_t \rho = \nabla\cdot (\nabla \rho + \rho(1-\rho)\nabla V).
\end{align}
Chosing $V = K \ast \rho$, we end up with problem (P1).
\subsubsection*{DG Methods}
Both problems are degenerate parabolic and may thus easily become convection domminated. Furthermore, in some applications, we are interested in the case of small diffusion, e.g. $D_1 = D_2 = \eps << 1$. This motivates / might motivate the use of discontinuous Galerkin methods.
\subsection*{Boundary conditions and Kernels $K$}
Depending on the application, the equations above can be supplemented with different types of boundary conditions: 
\begin{itemize}
 \item\textbf{No flux} We ask for the total flux in normal direction to vanish at the boundary - this, in particular, implies the conservation of mass:
 \begin{align*}
  \nabla \cdot (D_1(1-\rho)\nabla r + r \nabla \rho + r(1-\rho)\nabla V)\cdot n &= 0\text{ on } \partial\Omega,\\
  \nabla \cdot (D_2(1-\rho)\nabla b + b \nabla \rho + r(1-\rho)\nabla V)\cdot n &= 0\text{ on } \partial\Omega.
 \end{align*}
 \item\textbf{Flux conditions} Here we prescribe the flux at the boundary, in a non-linear fashon depending on the densities themselves. To this end, we also divide the boundary into a influx part $\Gamma\subset \partial\Omega$ and a outflux part and $\Sigma \subset\partial\Omega$ and abbreviate the respective fluxes:
 \begin{align*}
  j_r &:= D_1(1-\rho)\nabla r + r \nabla \rho + r(1-\rho)\nabla V,\\
  j_b &:= D_2(1-\rho)\nabla b + b \nabla \rho + r(1-\rho)\nabla V.
 \end{align*}
 Then, the b.c.s ware given as 
 \begin{align}
  j_r\cdot n &= \alpha_1 (1-\rho), &j_b\cdot n = \alpha_2 (1-\rho)\;\text{ on }\Gamma,\\
  j_r\cdot n &= \beta_1r, &j_b\cdot n = \beta_2 b\;\text{ on }\Gamma.
 \end{align}
 For the equation \eqref{eq:porousfem}, the situation simplifies to
 \begin{align}
  j\cdot n = \alpha (1-\rho)\;\text{ on }\Gamma,\\
  j\cdot n = \beta \rho\;\text{ on }\Gamma.
 \end{align}
\item\textbf{Other boundary conditions} In some applications (e.g. ion channels, nanopores) it makes also sense to consider mixed b.c.s consisten of Dirichlet on one part of the domain and no-flux on the other.
 \end{itemize}
Regarding the Kernels $K$, we have
\begin{itemize}
\item\textbf{Singular Kernels} One classic example would be the Newtonian Potential, i.e. $K$ such that $K\ast f$ means solving the Problem
\begin{align*}
 -\Delta u = f,
\end{align*}
with appropriate boundary conditions.
\item\textbf{Smooth Kernels} Kernels which still have full support but are smooth, radially symmetric and decreasing, e.g.
\begin{align*}
 K(x) = k(|x|),\; k(r) = \frac{1}{2}e^{-r^2}.
\end{align*}
\item\textbf{Kernels with compact support} These are Kernels, not necessarily smooth, e.g. of the form
\begin{align*}
 K(x) = \max\, \{ 1-x, 0\}
\end{align*}
\end{itemize}

\subsection*{Entropy structure}
We introduce the \emph{entropy functionals}
\begin{align}\label{eq:Eporous}
 F (\rho)=\int_{\Omega} \rho^m + \rho (K\ast \rho)\;dx,
\end{align}
for equation (P1) and 
\begin{align}\label{eq:E}
 E (r,\,b)=\int_{\Omega} r\log r + b\log b + (1-\rho)\log (1-\rho) + (r+b)V\;dx,
\end{align}
for equation (P2) respectively. These functionals are very important for the analyis and are in fact Lyaponov-functionals for the evolution - i.e. they are decreasing in time. 
% 
% Introducing the nonlinear diagonal mobility tensor ${\bf M}(r,b)$ given by
% \begin{equation*}
% {\bf M}(r,b) = \left(\begin{matrix}
% r(1-\rho) & 0 \\ 0 & Db (1-\rho)
% \end{matrix}\right)
% \end{equation*}
% we can rewrite \eqref{eq:baerbel} as the following formal gradient flow with respect to $E$
% \begin{equation}
% 	\partial_t \left( \begin{array}{l} r \\ b \end{array} \right) = \nabla \cdot
% \left({\bf M}(r,b) {\bf :} \nabla \left( \begin{array}{l} \partial_r E(r,b) \\ \partial_b E(r,b) \end{array} \right)\right). \label{eq:gradflow0}
% \end{equation}
% Inserting the definition \eqref{eq:E} of $E$, we recover \eqref{eq:baerbel}. 
For example, taking the time derivative of $E$ with respect to time, we see that
\begin{align}\label{eq:entdiss}
 \frac{d}{dt}E(r,b) = - \int_\Omega r(1-\rho)|\nabla u|^2 + b(1-\rho)|\nabla v|^2\;dx =: DE(r,b) \le 0,
\end{align}
where $u$ and $v$ are the so-called \emph{entropy variables} given by 
\begin{align}
  u := \partial_r E  &= \log r - \log(1-\rho) + V, \label{eq:etar} \\
  v := \partial_b E  &= \log b - \log(1-\rho) + W. \label{eq:etab}.
\end{align}
Integrating equation \eqref{eq:entdiss} w.r.t time, we obtain
\begin{align*}
 E(r,b) + \int_0^T DE(r,b)\;dt = E(r_0,b_0).
\end{align*}
% This is the crucial a priori estimate used in the existence analysis - \rr{Can be use this for the numerical analysis}
\textbf{This is an important feature that should be reflected by the numerical discretization}.

\subsection*{Work packages}
\begin{itemize}
 \item\textbf{(Local) DG Scheme for (P1)} Implement a (local) DG Scheme for (P1), e.g. extending the one of \cite{}. The scheme should for for all $m \in [1,\infty)$ (for $m>2$, the solution is only Lipschitz cont' at the boundary of its support). Interesting experiments would be the convergence to steady states, in particular rates of convergence.\\
 \rr{Christoph: Ich denke,  man sollte relativ easy das Paper \cite{Zhang2009} auf diesen Fall erweitern koennen. Wenn man zusaetzlich noch zeigen kann, dass das Schema die Entropiedissipation erhaelt + numerische Beispiele koennte rel. schnell ein nettes Paper werden. Die Entropie-geschichte wird wurde schon fuer Finite-Volume-Schemes gezeigt (\cite{Bessemoulin2012,Carrillo_Chertock_Huang_2015}, kannst du einschaetzen, und und wie einfach man das auf (hoehere Ordnung) DG uebertragen kann? Alternativ waere es auch eine Idee, die sachen von \cite{Carrillo_Chertock_Huang_2015} auf den Systemfall (i.e. (P2)) zu erweitern.}
 
 \item\textbf{(Local) DG Scheme for (P2)} Same as above, also with all the different boundary conditions. \rr{In fact, I already implemented a SIP-Scheme in Fenics where I treated the diffusion terms in a semi-explicit way, i.e.
 $$
 \nabla\cdot ((1-\rho^n)\nabla r^{n+1} + r^n\nabla \rho^{n+1}).
 $$
 This actually worked quite well. Can we analyze this, also already in the above case related to (P1) ?}
 
 \item\textbf{Combinations} Use the schemes from above to solve (P2) but with potential given as in (P1), i.e. $V=c_{11} K\ast r + c_{12}K\ast b$, $W=c_{22} K\ast b + c_{21}K\ast r$, \rr{see our recent preprint \cite{Berendsen2016} for details}.
 
\end{itemize}

% \subsection*{Optimal control problem}
% \rr{Wirklich ?}

\bibliographystyle{plain}
\bibliography{crossdg}


\end{document}
